% basics
\usepackage[utf8]{inputenc}
\usepackage[T1]{fontenc}
\usepackage{textcomp}
% \usepackage[dutch]{babel}
\usepackage{url}
\usepackage{hyperref}
\hypersetup{
 colorlinks,
 linkcolor={black},
 citecolor={black},
 urlcolor={blue!80!black}
}
\usepackage{graphicx}
\usepackage{float}
\usepackage{booktabs}
%\usepackage{enumitem}
\usepackage{enumerate}% http://ctan.org/pkg/enumerate
% \usepackage{parskip}
\usepackage{emptypage}
\usepackage{subcaption}
\usepackage{multicol}
\usepackage[usenames,dvipsnames]{xcolor}

\usepackage{pythonhighlight}

%\usepackage{cmbright}
%\usepackage{lmodern}

%\usepackage[familydefault,light]{Chivo} %% Option 'familydefault' only if the base font of the document is to be sans serif
%\usepackage[T1]{fontenc}

%\usepackage{libertinus}

%\usepackage[sfdefault]{overlock} %% Option 'sfdefault' only if the base font of the document is to be sans serif
%\usepackage[T1]{fontenc}


%\usepackage[scaled]{helvet}
%\renewcommand*\familydefault{\sfdefault} %% Only if the base font of the document is to be sans serif
%\usepackage[T1]{fontenc}

%\renewcommand*\sfdefault{lmssq}
%\renewcommand*\familydefault{\sfdefault} %% Only if the base font of the document is to be sans serif
%\usepackage[T1]{fontenc}

\usepackage[sfdefault,light]{inter} %% Option 'sfdefault' only if 'inter' is to be
% the base font of the document
\usepackage[T1]{fontenc}


\usepackage{amsmath, amsfonts, mathtools, amsthm, amssymb}
\usepackage{mathrsfs}
\usepackage{cancel}
\usepackage{bm}
\newcommand\N{\ensuremath{\mathbb{N}}}
\newcommand\R{\ensuremath{\mathbb{R}}}
\newcommand\Z{\ensuremath{\mathbb{Z}}}
\renewcommand\O{\ensuremath{\emptyset}}
\newcommand\Q{\ensuremath{\mathbb{Q}}}
\newcommand\C{\ensuremath{\mathbb{C}}}
\DeclareMathOperator{\sgn}{sgn}
\usepackage{systeme}
\let\svlim\lim\def\lim{\svlim\limits}
\let\implies\Rightarrow
\let\impliedby\Leftarrow
\let\iff\Leftrightarrow
\let\epsilon\varepsilon
\usepackage{stmaryrd} % for \lightning
\newcommand\contra{\scalebox{1.1}{$\lightning$}}
% \let\phi\varphi

\usepackage{geometry}




% correct
\definecolor{correct}{HTML}{009900}
\newcommand\correct[2]{\ensuremath{\:}{\color{red}{#1}}\ensuremath{\to }{\color{correct}{#2}}\ensuremath{\:}}
\newcommand\green[1]{{\color{correct}{#1}}}



% horizontal rule
\newcommand\hr{
    \noindent\rule[0.5ex]{\linewidth}{0.5pt}
}


% hide parts
\newcommand\hide[1]{}



% si unitx
\usepackage{siunitx}
\sisetup{locale = FR}
% \renewcommand\vec[1]{\mathbf{#1}}
\newcommand\mat[1]{\mathbf{#1}}


% tikz
\usepackage{tikz}
\usepackage{tikz-cd}
\usetikzlibrary{intersections, angles, quotes, calc, positioning}
\usetikzlibrary{arrows.meta}
\usepackage{pgfplots}
\pgfplotsset{compat=1.13}


\tikzset{
    force/.style={thick, {Circle[length=2pt]}-stealth, shorten <=-1pt}
}

% define some special colors for our palette 
\definecolor{p1_violet}{RGB}{207, 186, 240}
\definecolor{p1_dark_pink}{RGB}{255, 207, 210}
\definecolor{p1_light_blue}{RGB}{144, 219, 244}
\definecolor{p1_sky_blue}{RGB}{142, 236, 245}
\definecolor{p1_lime_green}{RGB}{185, 251, 192}
\definecolor{p1_teal}{RGB}{152, 245, 225}
\definecolor{p1_yellow}{RGB}{251, 248, 204}

\definecolor{p1_muted_green}{RGB}{216, 226, 220}
\definecolor{p1_muted_pink}{RGB}{254, 197, 187}
\definecolor{p1_muted_light_pink}{RGB}{250, 225, 221}
\definecolor{p1_muted_orange}{RGB}{254, 200, 154}
\definecolor{p1_muted_gray}{RGB}{248, 237, 235}

% theorems
\makeatother
\usepackage{thmtools}
\usepackage[framemethod=TikZ]{mdframed}
\mdfsetup{skipabove=1em,skipbelow=0em}


\theoremstyle{definition}

\declaretheoremstyle[
    spaceabove=0mm,
    headfont=\bfseries\sffamily\color{p1_sky_blue!50!black}, bodyfont=\normalfont,
    mdframed={
        linewidth=0.5pt,
        rightline=false, topline=false, bottomline=false,
        linecolor=p1_sky_blue!50!black, backgroundcolor=p1_sky_blue!20,
    }
]{thmgreenbox}

\declaretheoremstyle[
    spaceabove=0mm,
    headfont=\bfseries\sffamily\color{p1_light_blue!50!black}, bodyfont=\normalfont,
    mdframed={
        linewidth=0.5pt,
        rightline=false, topline=false, bottomline=false,
        linecolor=p1_light_blue!60!black, backgroundcolor=p1_light_blue!25,
    }
]{thmbluebox}


\declaretheoremstyle[
    spaceabove=0mm,
    headfont=\bfseries\sffamily\color{p1_muted_green!40!black}, bodyfont=\normalfont,
    mdframed={
        linewidth=0.5pt,
        rightline=false, topline=false, bottomline=false,
        linecolor=p1_muted_green!40!black, backgroundcolor=p1_muted_green!40
    }
]{thmblueline}

\declaretheoremstyle[
    spaceabove=0mm,
    headfont=\bfseries\sffamily\color{p1_muted_pink!40!black}, bodyfont=\normalfont,
    mdframed={
        linewidth=0.5pt,
        rightline=false, topline=false, bottomline=false,
        linecolor=p1_muted_pink!40!black, backgroundcolor=p1_muted_pink!30
    }
]{thmmutedpink}

\declaretheoremstyle[
    spaceabove=0mm,
    headfont=\bfseries\sffamily\color{p1_muted_gray!40!black}, bodyfont=\normalfont,
    mdframed={
        linewidth=0.5pt,
        rightline=false, topline=false, bottomline=false,
        linecolor=p1_muted_gray!40!black, backgroundcolor=p1_muted_gray!50!black!10
    }
]{thmmutedgray}


\declaretheoremstyle[
    spaceabove=0mm,
    headfont=\bfseries\sffamily\color{p1_violet!40!black}, bodyfont=\normalfont,
    mdframed={
        linewidth=0.5pt,
        rightline=false, topline=false, bottomline=false,
        linecolor=p1_violet!40!black, backgroundcolor=p1_violet!35,
    }
]{thmredbox}

\declaretheoremstyle[
    spaceabove=0mm,
    headfont=\bfseries\sffamily\color{p1_muted_pink!65!black}, bodyfont=\normalfont,
    numbered=no,
    mdframed={
        linewidth=0.5pt,
        rightline=false, topline=false, bottomline=false,
        linecolor=p1_muted_pink!65!black, backgroundcolor=p1_muted_pink!25,
    },
    qed=\qedsymbol
]{thmproofbox}

\declaretheoremstyle[
    spaceabove=0mm,
    headfont=\bfseries\sffamily\color{p1_teal!50!black}, bodyfont=\normalfont,
    numbered=no,
    mdframed={
        linewidth=0.5pt,
        rightline=false, topline=false, bottomline=false,
        linecolor=p1_teal!50!black, backgroundcolor=p1_teal!15,
    },
]{thmexplanationbox}

% \declaretheoremstyle[headfont=\bfseries\sffamily, bodyfont=\normalfont, mdframed={ nobreak } ]{thmgreenbox}
% \declaretheoremstyle[headfont=\bfseries\sffamily, bodyfont=\normalfont, mdframed={ nobreak } ]{thmredbox}
% \declaretheoremstyle[headfont=\bfseries\sffamily, bodyfont=\normalfont]{thmbluebox}
% \declaretheoremstyle[headfont=\bfseries\sffamily, bodyfont=\normalfont]{thmblueline}
% \declaretheoremstyle[headfont=\bfseries\sffamily, bodyfont=\normalfont, numbered=no, mdframed={ rightline=false, topline=false, bottomline=false, }, qed=\qedsymbol ]{thmproofbox}
% \declaretheoremstyle[headfont=\bfseries\sffamily, bodyfont=\normalfont, numbered=no, mdframed={ nobreak, rightline=false, topline=false, bottomline=false } ]{thmexplanationbox}

\declaretheorem[style=thmredbox, 
                name=Theorem,
                parent=chapter]{theorem1}
\newenvironment{theorem}{\begin{theorem1}}{\end{theorem1}}

\declaretheorem[style=thmredbox, 
                name=Theorem,
                numbered=no]{theorem3}
\newenvironment{theorem2}{\begin{theorem3}}{\end{theorem3}}

\declaretheorem[style=thmgreenbox,
                numberlike=theorem1, 
                name=Definition]{definition}
\declaretheorem[style=thmgreenbox, name=Definition, numbered=no]{definition2}

\declaretheorem[style=thmbluebox, 
                numberlike=theorem1,
                name=Example]{eg}
\declaretheorem[style=thmbluebox, 
                numbered=no,
                name=Example]{eg1}

\declaretheorem[style=thmredbox, 
                name=Proposition, 
                numberlike=theorem1]{prop}

\declaretheorem[style=thmredbox, 
                numberlike=theorem1,
                name=Lemma]{lemma}
\declaretheorem[style=thmredbox, name=Lemma, numbered = no]{lemma2}


\declaretheorem[style=thmredbox, numbered=no, name=Corollary]{corollary}



\declaretheorem[style=thmproofbox, name=Proof]{replacementproof}
\renewenvironment{proof}[1][\proofname]{\vspace{-10pt}\begin{replacementproof}}{\end{replacementproof}\vspace{1mm}}


\declaretheorem[style=thmexplanationbox, name=Proof]{tmpexplanation}
\newenvironment{explanation}[1][]{\vspace{-10pt}\begin{tmpexplanation}}{\end{tmpexplanation}\vspace{1mm}}

\declaretheorem[style=thmmutedgray, numbered=no, name=Q]{question}
\declaretheorem[style=thmgreenbox, numbered=no, name=Solution]{tmpsolution}
\newenvironment{solution}[1][]{\vspace{-11pt}\begin{tmpsolution}}{\end{tmpsolution}\vspace{1mm}}


\declaretheorem[style=thmblueline, numbered=no, name=Remark]{remark}
\declaretheorem[style=thmblueline, numbered=no, name=Note]{note}
\declaretheorem[style=thmmutedgray, numbered=no, name=Exercise]{exercisebox}


\newtheorem*{uovt}{UOVT}
\newtheorem*{notation}{Notation}
\newtheorem*{previouslyseen}{As previously seen}
\newtheorem*{problem}{Problem}
\newtheorem*{observe}{Observe}
\newtheorem*{property}{Property}
\newtheorem*{intuition}{Intuition}

\usepackage{etoolbox}
\AtEndEnvironment{vb}{\null\hfill$\diamond$}%
\AtEndEnvironment{intermezzo}{\null\hfill$\diamond$}%
% \AtEndEnvironment{opmerking}{\null\hfill$\diamond$}%

% http://tex.stackexchange.com/questions/22119/how-can-i-change-the-spacing-before-theorems-with-amsthm
\makeatletter
% \def\thm@space@setup{%
%   \thm@preskip=\parskip \thm@postskip=0pt
% }

\newcommand{\oefening}[1]{%
    \def\@oefening{#1}%
    \subsection*{Oefening #1}
}

\newcommand{\suboefening}[1]{%
    \subsubsection*{Oefening \@oefening.#1}
}

\newcommand{\exercise}[1]{%
    \def\@exercise{#1}%
    \subsection*{Exercise #1}
}

\newcommand{\subexercise}[1]{%
    \subsubsection*{Exercise \@exercise.#1}
}

\usepackage{xifthen}

\def\testdateparts#1{\dateparts#1\relax}
\def\dateparts#1 \relax{
    \marginpar{\small\textsf{\mbox{\textbf{#1}}}}
}

\def\@lesson{}%
\newcommand{\lesson}[3]{
    \ifthenelse{\isempty{#3}}{%
        \def\@lesson{Lecture #1}%
    }{%
        \def\@lesson{Lecture #1: #3}%
    }%
    \subsection*{\@lesson}
    \vspace{-5.10mm}
    \noindent\makebox[\linewidth]{\rule{\textwidth}{1pt}}
    \vspace{-12.50mm}
    {\begin{flushright}\small\textbf{Date:\quad#2}\end{flushright}}\par
    \vspace{0.20cm}
%    \testdateparts{#2}
}

% \renewcommand\date[1]{\marginpar{#1}}

% fancy headers
\usepackage{fancyhdr}
\pagestyle{fancy}

% \fancyhead[LE,RO]{Gilles Castel}
\fancyhead[RO,L]{\@lesson}
\fancyhead[R,LO]{}
\fancyfoot[L,RO]{\thepage}
\fancyfoot[C]{\leftmark}

\makeatother


% notes
\usepackage{todonotes}
\usepackage{tcolorbox}

\tcbuselibrary{breakable}
\newenvironment{verbetering}{\begin{tcolorbox}[
    arc=0mm,
    colback=white,
    colframe=green!60!black,
    title=Opmerking,
    fonttitle=\sffamily,
    breakable
]}{\end{tcolorbox}}

\newenvironment{noot}[1]{\begin{tcolorbox}[
    arc=0mm,
    colback=white,
    colframe=white!60!black,
    title=#1,
    fonttitle=\sffamily,
    breakable
]}{\end{tcolorbox}}

% figure support
\usepackage{import}
\usepackage{xifthen}
\pdfminorversion=7
\usepackage{pdfpages}
\usepackage{transparent}
\newcommand{\incfig}[1]{%
    \def\svgwidth{0.8\columnwidth}
    \import{./figures/}{#1.pdf_tex}
}

\newcommand{\executeiffilenewer}[3]{%
\ifnum\pdfstrcmp{\pdffilemoddate{#1}}%
{\pdffilemoddate{#2}}>0%
{\immediate\write18{#3}}\fi%
}
\newcommand{\includesvg}[1]{%
\executeiffilenewer{#1.svg}{#1.pdf}%
{inkscape -z -D --file=#1.svg %
--export-pdf=#1.pdf --export-latex}%
\input{#1.pdf_tex}%
}

% %http://tex.stackexchange.com/questions/76273/multiple-pdfs-with-page-group-included-in-a-single-page-warning
\pdfsuppresswarningpagegroup=1

\usepackage{titlesec}    
\titleformat
{\chapter} % command to format
[block] % shape: hang, display, block, frame etc
{\sffamily\bfseries\LARGE}  % format of label + chapter title
{\chaptertitlename\ \thechapter:} % label "Chapter 1:"
{1.5ex} % separation label - chapter title
{} % code before

\titlespacing{\chapter}
{0pt} %left of the label + title
{5mm} % vertical space before the title
{5mm} % idem after title (in ex units + glue)

\titleformat{\section}
[block] % shape 
{\sffamily\bfseries\Large}% format (keep the chapter font family!)
{\thesection.} % label "1.1."
{2.8ex}% separation label -  section title 
{}

\titlespacing{\section}
{0pt} %left of label + section title
{*4} % before the label + section title
{*1.5} % after


\usepackage{silence}
\WarningFilter{mdframed}{You got a bad break}
\WarningFilter{hyperref}{old toc file detected}
\hfuzz=1000pt
\tolerance=20000
\hbadness=19999

\DeclarePairedDelimiter{\ceil}{\lceil}{\rceil}
\DeclarePairedDelimiter{\floor}{\lfloor}{\rfloor}
